% !TEX root = ./amsa_main.tex
\section{Primal-dual Coordinate Friendly Operators}\label{sec:p-d}
{We study how to solve the  problem
\begin{equation}
\underset{x\in\mathbb{H}}{\text{\normalfont minimize }} f(x)+g(x)+h(Ax),\label{pdproblem}
\end{equation}
{with primal-dual splitting algorithms, as well as their coordinate update versions. Here, $f$ is differentiable %, $g$ and $h$ are possibly nondifferentiable,
and $A$ is a ``$p$-by-$m$" linear operator from $\mathbb{H}=\HH_1\times\cdots\times\HH_m$ to $\mathbb{G}=\GG_1\times\cdots\times\GG_p$.
%$\Box$ is the infimal convolution operator defined as $(h\Box l)(y)=\inf_{z\in\GG} h(z)+l(y-z)$.
\cut{[do we mention saddle-point and VI at the end of this section?]} Problem~\eqref{pdproblem} abstracts many applications in image processing and machine learning.
\begin{example}[image deblurring/denoising]
Let $u^0$ be an image, where $u_i^0\in[0,255]$, and $B$ be the blurring linear operator. Let $\|\nabla u\|_1$ be the anisotropic\footnote{Generalization to the isotropic case is straightforward by grouping variables properly.} total variation of $u$ (see \eqref{eqn:tv_def} for definition). Suppose that $b$ is a noisy observation of $Bu^0$. Then, we can try to recover $u^0$ by solving
\begin{equation}
\Min_u\, \frac{1}{2}\|Bu-b\|^2+\iota_{[0,255]}(u)+\lambda\|\nabla u\|_1,
\end{equation}
which can be written in the form of $\eqref{pdproblem}$ with $f=\frac{1}{2}\|B\cdot-b\|^2$, $g=\iota_{[0,255]}$, $A=\nabla$, and $h=\lambda\|\cdot\|_{1}$.
\end{example}
More examples with formulation \eqref{pdproblem} will be given in \S\ref{sec:emp}. %Definitions and examples are given in the subsection $\ref{sec:emp}$.
In general, primal-dual methods are capable of solving complicated problems involving constraints and the compositions of proximable and linear maps like $\|\nabla u\|_1$.

In many applications, although $h$ is  proximable, $h\circ A$  is generally non-proximable and non-differentiable. To avoid using slow subgradient methods,  we can consider the  primal-dual splitting approaches to separate $h$ and $A$ so that $\prox_{h}$ can be applied.
%From the given examples, we can see in many applications $h$ is not differentiable. In that case, it is also hard to evaluate the proximal operator of $h\circ A$. We need to find a way to separate $h$ and $A$.
\cut{
The first approach is \emph{variable splitting}: first rewrite the problem~\eqref{pdproblem} as $$\Min_{x,y}\, f(x)+g(x)+h(y),\quad \St~ y=Ax,$$ and then apply ADMM~\eqref{eq:admmx-y}. The $y$-subproblem of ADMM reduces to computing $\prox_{h}$. The $x$-subproblem of ADMM has the form
\beq\label{admmx}
\Min_x f(x)+g(x)+\frac{\eta}{2}\|Ax-y\|^2+(\mbox{linear terms in }x).
\eeq
When $f+g$ is differentiable or proximable, \eqref{admmx} can be solved by an iterative procedure. In image deblurring, with $g=0$ and proper boundary conditions, even a closed-form solution can be found~\cite{wang2008new}. Generally, it is not easy to directly solve \eqref{admmx}. By introducing more auxiliary variables, $A$ can also be separated from $g$ and $h$, but the resulting subproblem involving $A$ will need form and invert  $A^\top A$~\cite{o2014primal}. (A remedy is to linearize $\|Ax-y\|^2$ in the subproblem, yet it can be shown as a special case of the primal-dual splitting approach below.)


The second approach is \emph{primal-dual splitting}. %, based on upd and the Fenchel biconjugation.
}
We derive that the equivalent form (for convex case) of $\eqref{pdproblem}$ is to find $x$ such that
\begin{equation}
0\in (\nabla f+\partial g+A^\top\circ\partial h\circ A)(x).
\end{equation}
%where $\partial h\Box\partial l=(\partial h^{-1}+\partial l^{-1})^{-1}$.
Introducing the dual variable $s\in\mathbb{G}$ and applying the biconjugation property:  $s\in \partial h(Ax)\Leftrightarrow Ax\in \partial h^*(s)$, yields the equivalent condition
%and use the property $s\in \partial h\Box\partial l)\circ Ax\Leftrightarrow Ax\in \partial h^*(s)+\partial l^*(s)$ and $l^*$ is %differentiable
\begin{equation}
0\in\bigg(\underbrace{\begin{bmatrix}
\nabla f\\
0
\end{bmatrix}}_{\mbox{operator}~\cA}+\underbrace{
\begin{bmatrix}
\partial g\\
\partial h^*
\end{bmatrix}+\begin{bmatrix}
0&A^\top\\
-A&0
\end{bmatrix}}_{\mbox{operator}~\cB}\bigg) \underbrace{\begin{bmatrix}
x\\
s
\end{bmatrix}}_{z},\label{pdkkt}
\end{equation}
%If we define $z=\begin{bmatrix}x\\ s\end{bmatrix}$, the above condition can be written as $0\in\cA z+\cB z$.
which we shorten as $0\in\cA z+\cB z$, with $z\in\HH\times\GG=:\FF$.

\cut{\begin{remark}
In a variant of problem $\eqref{pdproblem}$\cite{}, $h(Ax)$ is replaced by $(h\Box l)(Ax)$ with $l$ a strongly convex function (thus $l^*$ is differentiable). $\Box$ is the infimal convolution operator defined as $(h\Box l)(y)=\inf_{z\in\GG} h(z)+l(y-z)$, which is used, for example, in dual smoothing\cite{?}. Because $s\in (\partial h\Box\partial l)(Ax)\Leftrightarrow Ax\in \partial h^*(s)+\partial l^*(s)$, $\eqref{pdkkt}$ becomes: $0\in\begin{bmatrix}
\nabla f(x)\\
0
\end{bmatrix}+
\begin{bmatrix}
\partial g(x)\\
\partial h^*(s)
\end{bmatrix}+\begin{bmatrix}
0&A^\top\\
-A&0
\end{bmatrix}\begin{bmatrix}
x\\
s
\end{bmatrix}$, which can be solved in the same way as $\eqref{pdproblem}$.
\end{remark}
}
%The monotone inclusion problem $\eqref{pdkkt}$ can be solved by introducing a metric
%$$M=\begin{bmatrix}
%\frac{1}{\eta}I&A^\top\\
%A&\frac{1}{\gamma}I
%\end{bmatrix}\succ 0$$
%and applying a forward backward splitting algorithm: $z^{k+1}=\cT z^k=\cJ_{M^{-1}\cB}\circ(\cI-M^{-1}\cA)z$. The V\~{u}-Condat algorithm will be recovered:
Problem $\eqref{pdkkt}$ can be solved  by the Condat-V\~{u} algorithm~\cite{condat2013primal, vu2013splitting}:
\begin{equation}
\left\{
\begin{array}{l}
s^{k+1}=\prox_{\gamma h^*} (s^k+\gamma Ax^k),\\
x^{k+1}=\prox_{\eta g}(x^k-\eta(\nabla f(x^k)+A^\top(2s^{k+1}-s^k))),
\end{array}
\right.\label{vucondat}
\end{equation}
%In particular, the algorithm applied to the extended monotropic programming $\eqref{emp}$ is
%\begin{equation}
%\left\{
%\begin{array}{l}
%s^{k+1}=s^k+\gamma (Ax^k-b)\\
%x^{k+1}=\prox_{\eta g}(x^k-\eta(\nabla f(x^k)+A^\top(2s^{k+1}-s^k)))
%\end{array}
%\right.\label{vucondatemp}
%\end{equation}
which explicitly applies $A$ and $A^\top$ and updates $s,x$ in a Gauss-Seidel style~\footnote{By the Moreau identity: $\prox_{\gamma h^*}=\cI -\gamma \prox_{\frac{1}{\gamma}h}(\frac{\cdot}{\gamma})$, one can compute $\prox_{\frac{1}{\gamma} h}$ instead of $\prox_{\gamma h^*}$, which inherits the same separability properties from $\prox_{\frac{1}{\gamma} h}$.}.
We introduce an operator $\TVC:\FF\to \FF$ and write
$$\mbox{iteration \eqref{vucondat}}\quad\Longleftrightarrow\quad z^{k+1}=\TVC(z^k).$$

%\begin{remark}

Switching the orders of  $x$ and $s$ yields the following algorithm: %, $x$ and $s$ are in equivalent positions, we
% updating order by changing the metric to $M=\begin{bmatrix}
%%\frac{1}{\eta}I&-A^\top\\
%%-A&\frac{1}{\gamma}I
%%\end{bmatrix}$,
%have another algorithm similar to $\eqref{vucondat}$:
\begin{equation}
\left\{
\begin{array}{l}
x^{k+1}=\prox_{\eta g}(x^k-\eta(\nabla f(x^k)+A^\top s^k)),\\
s^{k+1}=\prox_{\gamma h^*} (s^k+\gamma A(2x^{k+1}-x^k)),
\end{array}
\right.{\text{ as }z^{k+1}=\cT'_{\textnormal{CV}} z^k.}\label{vucondat2}
\end{equation}
%\end{remark}
It is known from \cite{combettes2014forward,davis2014convergence} that both $\eqref{vucondat}$ and $\eqref{vucondat2}$  reduce to iterations of nonexpansive operators (under a special metric), i.e., $\TVC$ is  nonexpansive; see Appendix~\ref{sec:vc-op} for the reasoning. %For simplicity, our discussion below only focuses on \eqref{vucondat}.
\begin{remark}
Similar primal-dual algorithms can be used to solve other problems such as saddle point problems~\cite{lebedev1967duality,mclinden1974extension,briceno2013monotone} and variational inequalities~\cite{tseng1991applications}. Our coordinate update algorithms below apply to these problems as well.
\end{remark}
\subsection{Primal-dual Coordinate Update Algorithms}\label{sec:pdcu}
In this subsection, we make the following assumption.
\begin{assumption}
Functions $g$ and $h^*$ in the problem~\eqref{pdproblem} are separable and proximable. Specifically, $$g(x)=\displaystyle\sum_{i=1} ^m g_i(x_i)\quad\mbox{and}\quad h^*(y)=\displaystyle\sum_{j=1}^ph^*_i(y_i).$$\label{pdassum}
Furthermore, $\nabla f$ is CF.
\end{assumption}
%We show that either $\TVC$ or $\cT'_{\textnormal{VC}}$ is CF under this assumption.
\begin{proposition}\DIFaddbegin \label{prop1}
\DIFaddend Under Assumption $\ref{pdassum}$, the followings hold:
\begin{enumerate}[(a)]
\item when $p=O(m)$, the Condat-Vu operator $\TVC$ in \eqref{vucondat} is CF, more specifically, $$\nops{\{z^k,Ax\}}{\{z^+,Ax^+\}}=O\left(\frac{1}{m+p}\nops{z^k}{\TVC z^k}\right);$$
\item when $m\ll p$ and $\nops{x}{\nabla f(x)}=O(m)$, the Condat-Vu operator $\cT'_\textnormal{CV}$ in \eqref{vucondat2} is CF, more specifically, $$\nops{\{z^k,A^\top s\}}{\{z^+,A^\top s^+\}}=O\left(\frac{1}{m+p}\nops{z^k}{\cT'_{\textnormal{CV}} z^k}\right).$$
\end{enumerate}
\end{proposition}
%To this end, we compare $\nops{z^k}{\TVC z^k}$ or $\nops{z^k}{\cT'_{\textnormal{VC}}z^k}$ with $\nops{\{z^k,\cM(z^k)\}}{\{z^+,\cM(z^+)\}}$, where $z^+_j=(\TVC z^k)_j$ (or $(\cT'_{\textnormal{VC}} z^k)_j$) and $z^+_i=z^k_i$ for $i\neq j$. \cut{(derive the numbers in Yangyang's new definition.)}

 %, we show that (the operator underlying) the iteration \eqref{vucondat} is CF. %discuss how to perform coordinate updates based on \eqref{vucondat}.
%Recall that, in this section, we are interested in the conditions that make $\cT_1$ and $\cT_2$ CF. %We need to compare $\mathfrak{M}[z\to(\cT z)_i]$ with $\mathfrak{M}[z\to\cT z]$. As the last operators we apply when computing $(\cT z)_i$, we need $\prox_{\gamma h^*}$ and $\prox_{\eta g}$ to be separable(or $\cC_2$ is enough?). As intermediate steps, $\nabla f$ needs to be easy to update/compute. We cache $\nabla f(x)$ together with $Ax$ and $A^\top s$ at each iteration and refresh them at a low cost when one or a few coordinates of $z$ are changed using $(\cT z)_i$
%When we want to compute some coordinates of $\cT z^k$, we will need to compute some coordinates of $\prox_{\eta g}$ and $\prox_{\gamma h^*}$, so we will need them to be BC-friendly. After updating some coordinates each time, we can update $\nabla f(x)$, $\nabla l^*(s)$, $Ax$, $A^Ts$ and cache them, so we only need $\nabla f(x)$ and $\nabla l^*(s)$ to be easy-to-update.We make the following assumption:
\cut{
\begin{assumption}
$g(x)=\displaystyle\sum_{i=1} ^m g_i(x_i)$ and $h^*(y)=\displaystyle\sum_{j=1}^ph^*_j(y_j)$ where the $g_i$'s and $h^*_j$'s are proximable functions, i.e. $\prox_g,\prox_{h^*}\in\cC_1$ (can be relaxed to $\cC_2$). $\nabla f$ (and $\nabla l^*$) are easy-to-update.\label{pdassum}
\end{assumption}

We show under this assumption either $\cT_1$ or $\cT_2$ is CF. }
% Due to the nature of problem $\eqref{pdproblem}$ and algorithm $\eqref{vucondat}$, the variables $x$ and $s$ are coupled together by the matrix $A$. In order to study this coupling we make some definitions here. We define the set $\JJ(i)\subset \{1,2,\cdots, p\}$ s.t. $A^\top_{i,j}\neq 0,\forall j\in\JJ(i)$, and $\II(j)\subset\{1,2,\ldots,m\}$ s.t. $A_{j,i}\neq 0,\forall i\in \II(j)$. We also define $m_j=|\II(j)|$.
%We can now illustrate the coordinate update scheme further and verify it is BC-friendly under the above assumption.
\begin{proof}
Computing $z^{k+1}=\TVC z^k$ involves evaluating $\nabla f$, $\prox_g$, and $\prox_{h^*}$, applying $A$ and $A^\top$, and adding vectors.
%$Ax^k$, with $O(mp)$ operations;
%$\prox_{\gamma h^*}$, with $O(p)$ operations;
%$A^\top (2s^{k+1}-s^k)$, with another $O(mp)$ operations;
%$\nabla f(x^k)$, with $\mathfrak{M}[x\to\nabla f(x)]$ operations;
%$\prox_{\eta g}$, with $O(m)$ operations.
It is easy to see  $\nops{z^k}{\TVC z^k}=O(mp+m+p)+\mathfrak{M}[x\to\nabla f(x)]$, and $\nops{z^k}{\cT'_{\textnormal{CV}}z^k}$ is the same.\\
%When $m\gg p$ or $m\sim p$, $\TVC$ is CF, provided $\prox_g\in\cC_2,\prox_{h^*}\in\cF_1,\nabla f\in\cF$ and $\nops{t}{(\prox_{h^*})_i}\ll m$.
(a) We assume $\nabla f\in\cF_1$ for simplicity, and other cases are similar.
%The reason to maintain $Ax$ is that in order to compute $x^{k+1}_i$, we need $(A^\top s^{k+1})_i$, thus $s^{k+1}$, and therefore the entire $Ax^k$. To avoid computing $Ax$ repeatedly, we cache it and update it accordingly.
\begin{enumerate}
\item If $(\TVC z^k)_j=s^{k+1}_i$, computing it involves: adding $s^k_i$ and $\gamma (Ax^k)_i$, and evaluating $\prox_{\gamma h^*_i}$. In this case $\nops{\{z^k,Ax\}}{\{z^+,Ax^+\}}=O(1)$.
%Following this route, computing $s^{k+1}_i$ involves only evaluating $\prox_{\gamma h^*_i}$ from cached quantities, which costs $O(1)$ operations.
\item If $(\TVC z^k)_j=x^{k+1}_i$, computing it involves evaluating: the entire $s^{k+1}$ for $O(p)$ operations, $(A^\top(2s^{k+1}-s^k))_i$ for $O(p)$ operations, $\prox_{\eta g_i}$ for $O(1)$ operations, $\nabla_i f({x}^{k})$ for $O(\frac{1}{m}\nops{x}{\nabla f(x)})$ operations, as well as updating $Ax^+$ for $O(p)$ operations.
In this case\\ $\nops{\{z^k,Ax\}}{\{z^+,Ax^+\}}=O(p+\frac{1}{m}\nops{x}{\nabla f(x)})$.
\end{enumerate}
Therefore, $\nops{\{z^k,Ax\}}{\{z^+,Ax^+\}}=O\big(\frac{1}{m+p}\nops{z^k}{\TVC z^k}\big)$. %unless $m\ll p$.
\\[5pt]
(b) When $m\ll p$ and $\nops{x}{\nabla f(x)}=O(m)$, following  arguments similar to the above, we have\\$\nops{\{z^k,A^\top s\}}{\{z^+,A^\top s^+\}}=O(1)+\nops{x}{\nabla_i f(x)}$ if $(\cT'_{\textnormal{CV}} z^k)_j=x_i^{k+1}$; and $\nops{\{z^k,A^\top s\}}{\{z^+,A^\top s^+\}}=O(m)+\nops{x}{\nabla f(x)}$ if $(\cT'_{\textnormal{CV}} z^k)_j=s_i^{k+1}$.\\ In both cases $\nops{\{z^k,A^\top s\}}{\{z^+,A^\top s^+\}}=O(\frac{1}{m+p}\nops{z^k}{\cT'_{\textnormal{CV}} z^k})$.
\end{proof}
%Comparing $O(p)+\mathfrak{M}[\{x^k,\nabla f(x^k),(i)\}\to \nabla f(\hat{x}^{k+1})]$ with $O(mp+m+p)+\mathfrak{M}[x\to\nabla f(x)]$, we can claim the operator $\cT_1$ is CF, unless $m\ll p$. In case $m\ll p$, we can prove the operator $\cT_2$ is CF, since by using $\eqref{vucondat2}$ and caching $\nabla f(x)$ and $A^\top s$, computing $x_i^{k+1}$  costs $O(1)+\mathfrak{M}[\{x^k,\nabla f(x^k),(i)\}\to \nabla f(\hat{x}^{k+1})]$ operations and computing $s_i^{k+1}$ costs $O(m)$ operations.
%\begin{remark}
%We may maintain $A^\top s^k$ and update it, instead of computing it directly every time. But it won't affect the average coordinate update cost. Since either way, updating the entire $\cT z$ coordinate by coordinate will involve computing (or updating)all the coordinates of $A^\top s$ from every (updated) coordinate of $s$. For the same reason, even when $A$ has some sparsity patterns, using algorithm $\eqref{vucondat}$ or $\eqref{vucondat2}$ to do coordinate updates will have the same cost on average. However, caching $Ax$ is always useful even when $A$ has some sparsity patterns, because it helps avoid repetitive computations when computing $Ax$.
%\end{remark}
%we first need to compute all the new $x_i$ and $s_j$ using the cached values, which costs $O(m+p)$ operations; then we need to refresh $Ax$, $A^\top s$ and $\nabla f(x)$, which costs $O(mp)+\mathfrak{M}[x\to\nabla f(x)]$ operations. So the total number of operations we need is $O((m+1)(p+1))+\mathfrak{M}[x\to\nabla f(x)]$. When we need to compute $(\cT z)_i$ and it is a coordinate $s_j$ of $s$, we need to compute $s_j$ and update $A^\top s$, which costs $O(m+1)$ operations. It is less then $\frac{1}{p+1}$ of the operations to compute $\cT z$.
%%When the coordinate chosen is a coordinate $x_i$ of $x$, we first need to update $s_j^{k+1}$ for every $j\in\JJ(i)$, which costs at most $O(p)$ operations; then we need to evaluate $A_i^\top s^{k+1}$, which costs another $O(p)$ operations at most; finally we need to compute $x_i^{k+1}$ and update $Ax^{k+1}$ and $\nabla f(x^{k+1})$, which costs $O(p+1)+\mathfrak{M}[\{x^k,\nabla f(x^k),(i)\}\to \nabla f(x^{k+1})]$ operations. So the total cost is $O(3p+1)+\mathfrak{M}[\{x^k,\nabla f(x^k),(i)\}\to \nabla f(x^{k+1})]$. Because $\nabla f$ is easy-to-update, we can see $\cT$ is BC-friendly if $m$ and $p$ are of the same order.
%%First we need $\prox_{\gamma h^*}$ and $\prox_{\eta g}$ to be BC-friendly or easy to update. We assume $g(x)=\displaystyle\sum_{i=1} ^m g_i(x_i)$ and $h^*(y)=\displaystyle\sum_{j=1}^ph^*_j(y_j)$. Secondly, we also need $\nabla f$ to be BC-friendly %and $A$ to be sparsely supported.
%
%
%For the first choice, when $x_i$ are selected to be updated, we compute the coordinates of $s$ we need but discard them after updating $x_i$:
%\begin{equation}
%\left\{
%\begin{array}{l}
%\text{when }x_i\text{ is chosen to be updated, compute}\\
%\tilde{s_j}^{k+1}=\prox_{\gamma h_j^*} (s_j^k+\gamma Ax^k),\forall j\in\JJ(i)\\
%\tilde{x_i}^{k+1}=\prox_{\eta g_i}(x_i^k-\eta(\nabla_i f(x^k)+\sum_{j\in\JJ(i)}A_{j,i}^\top(2\tilde{s}_j^{k+1}-s_j^k)))\\
%\text{then update }x_i^{k+1}=x_i^k+\eta_k(\tilde{x}_i^{k+1}-x_i^k)\text{ and discard all the }\tilde{s}_j^{k+1} \\
%\text{when }s_j\text{ is chosen to be updated, compute}\\
%\tilde{s_j}^{k+1}=\prox_{\gamma h_j^*} (s_j^k+\gamma Ax^k),\\
%\text{then update }s_j^{k+1}=s_j^k+\eta_k(\tilde{s}_j^{k+1}-s_j^k)
%\end{array}
%\right.
%\end{equation}
%
%The second choice is, instead of discarding the newly computed $s_j$ every time we update $x_i$, we keep the $s_j$ when we update a single one of the $x_i$'s with $i\in\II(j)$:
%\begin{equation}
%\left\{
%\begin{array}{l}
%\text{when }x_i\text{ is chosen to be updated, compute}\\
%\tilde{s_j}^{k+1}=\prox_{\gamma h_j^*} (s_j^k+\gamma Ax^k),\forall j\in\JJ(i)\\
%\tilde{x_i}^{k+1}=\prox_{\eta g_i}(x_i^k-\eta(\nabla_i f(x^k)+\sum_{j\in\JJ(i)}A_{j,i}^\top(2\tilde{s}_j^{k+1}-s_j^k)))\\
%\text{update }x_i^{k+1}=x_i^k+\eta_k(\tilde{x}_i^{k+1}-x_i^k) \\
%\text{update }s_j^{k+1}=s_j^k+\eta_k(\tilde{s}_j^{k+1}-s_j^k) \text{ if }s_j\text{ is assigned to }x_i\text{ and discard all the other}\tilde{s}_j^{k+1}\text{'s}
%\end{array}
%\right.
%\end{equation}
%The third choice is, we update $s_j$ whenever an $x_i$ with $i\in\II(j)$ is updated, but we need to scale the update with some constant:
%\begin{equation}
%\left\{
%\begin{array}{l}
%\text{when }x_i\text{ is chosen to be updated, compute}\\
%\tilde{s_j}^{k+1}=\prox_{\gamma h_j^*} (s_j^k+\gamma Ax^k),\forall j\in\JJ(i)\\
%\tilde{x_i}^{k+1}=\prox_{\eta g_i}(x_i^k-\eta(\nabla_i f(x^k)+\sum_{j\in\JJ(i)}A_{j,i}^\top(2\tilde{s}_j^{k+1}-s_j^k)))\\
%\text{update }x_i^{k+1}=x_i^k+\eta_k(\tilde{x}_i^{k+1}-x_i^k) \\
%\text{update }s_j^{k+1}=s_j^k+\frac{\eta_k}{m_j}(\tilde{s}_j^{k+1}-s_j^k),\forall j\in\JJ(i)
%\end{array}
%\right.\label{pd3rd}
%\end{equation}
%%Thus when we compute $s^{k+1}_j$, we need $s^k_j$ and $x^k_i$ for all $i\in \II(j)$; when we compute $x^{k+1}_i$, we need to compute $s^{k+1}_j$ first, for all $j\in \JJ(i)$, thus involving more components of $x$, and we also need to compute
%%$\nabla_i f(x^k)$. Note that when we compute the $s^{k+1}_j$'s, we don't actually update them. We can cache them for future use, depending on the actual type of algorithm we are running.
%
%%We can see under assumption $\ref{pdassum}$, the BC update version of $\eqref{vucondat}$ is still hard to implement.
%Note that, in the above coordinate update scheme, when we compute $x^{k+1}_i$, the $s^{k+1}$ we computed never get restored or cached, because they may never appear again in our coordinate updating process. This waste of computation is due to the nature of this algorithm. Caching $Ax$ and updating it whenever necessary is useful because it reduces the amount of the wasted computation. We will see this same idea in the next sections.
%where the $s$ update depends on the $x$ update. We can design coordinate updating algorithms based on this algorithm and it is favorable when $A$ has sparse rows in stead of sparse columns.
%\end{remark}
%Next we will study a special case where we can decouple $x$ and $s$.
\subsection{Extended Monotropic Programming}\label{sec:emp}
We develop a primal-dual coordinate update algorithm for the extended monotropic program:
\begin{equation}
\begin{array}{rl}
\underset{x\in\mathbb{H}}{\text{minimize  }} &~g_1(x_1)+g_2(x_2)+\cdots+g_m(x_m)+f(x),\\
\St &~A_1x_1+A_2x_2+\cdots+A_mx_m=b,
\end{array}\label{emp}
\end{equation}
where $x = (x_1, \ldots, x_m) \in \HH=\HH_1\times\ldots \times \HH_m$ with $\HH_i$ being Euclidean spaces. It generalizes linear, quadratic, second-order cone, semi-definite programs by allowing  extended-valued objective functions $g_i$ and $f$.
It is a special case of $\eqref{pdproblem}$ by letting $g(x)=\displaystyle\sum_{i=1}^m g_i(x_i)$, $A=[A_1,\cdots, A_m]$ and
$h=\iota_{\{b\}}$.
\begin{example}[quadratic programming]
Consider the quadratic program
\begin{equation}
\underset{x\in\mathbb{R}^m}{\textnormal{minimize }} \frac{1}{2}x^\top Ux+c^\top x,~\St~ Ax=b,~x\in X,\label{qp}
\end{equation}
where $U$ is a symmetric positive semidefinite matrix and $X=\{x:x_i \geq 0~\forall i\}$.
Then, \eqref{qp} is a special case of \eqref{emp}\cut{ $$\underset{x\in\mathbb{R}^n}{\textnormal{minimize }} \frac{1}{2}x^\top Ux+c^\top x+\iota_{X}(x)+\iota_{\{b\}}(Ax).$$}
with  $g_i(x_i)=\iota_{\cdot\geq 0}(x_i)$, $f(x)=\frac{1}{2}x^\top Ux+c^\top x$ and $h=\iota_{\{b\}}$.
\end{example}
\begin{example}[Second Order Cone Programming (SOCP)]
The SOCP
\begin{align*}
\underset{x\in\mathbb{R}^m}{\textnormal{minimize }} ~c^\top x,&\quad\St~ Ax=b,\\
&\hspace{56pt} x\in X=Q_1\times \cdots\times Q_{n},
\end{align*}
(where the number of cones $n$ may not be equal to the number of blocks $m$,) can be written in the form of \eqref{emp}: $\Min_{x\in\mathbb{R}^m} \iota_X(x)+c^\top x+\iota_{\{b\}}(Ax).$ %Details of SOCP can be found in section $\ref{sec:socp}$.
%and linear programming is a special case of this problem.
\end{example}
%Algorithm $\eqref{vucondat}$ applied to the extended monotropic programming is:
%\begin{equation}
%\left\{
%\begin{array}{l}
%s^{k+1}=s^k+\gamma (Ax^k-b)\\
%x^{k+1}=\prox_{\eta g}(x^k-\eta(\nabla f(x^k)+A^\top(2s^{k+1}-s^k)))
%\end{array}
%\right.
%\end{equation}
Applying iteration $\eqref{vucondat}$ to problem $\eqref{emp}$ and eliminating $s^{k+1}$ from the second row yield the Jacobi-style update (denoted as $\cT_\textnormal{emp}$):
\begin{equation}
\left\{
\begin{array}{l}
s^{k+1}=s^k+\gamma (Ax^k-b),\\
x^{k+1}=\prox_{\eta g}(x^k-\eta(\nabla f(x^k)+A^\top s^k+2\gamma A^\top Ax^k-2\gamma A^\top b)).
\end{array}
\right.\label{pdemp}
\end{equation}
To the best of our knowledge, this update is never found in the literature. Note that $x^{k+1}$ no longer depends on $s^{k+1}$, making it more convenient to perform coordinate updates.
%The coordinate updates of $\eqref{pdemp}$ have the same costs as those of $\eqref{vucondat}$, so \eqref{pdemp} is CF except when $m\ll p$. We do have $p<m$ since the linear constraints $Ax=b$ are generally under-determined.
%, updating one coordinate of $s$ still costs $O(1)$ operations and the cost of updating one coordinate of $x$ is still $O(p)+\mathfrak{M}[\{x^k,\nabla f(x^k),(i)\}\to \nabla f(\hat{x}^{k+1})]$ operations. The cost for computing the entire $\cT z$ coordinate by coordinate is $O(mp+p)+m\mathfrak{M}[\{x^k,\nabla f(x^k),(i)\}\to \nabla f(\hat{x}^{k+1})]$ operations.
%
%If we instead do not maintain $Ax$, updating one coordinate of $s$ will cost $O(m)$ operations and the cost of updating one coordinate (involving computing $(A^\top Ax)_i$) of $x$ will be $O(m+p)+\mathfrak{M}[\{x^k,\nabla f(x^k),(i)\}\to \nabla f(\hat{x}^{k+1})]$ operations. The cost for computing the entire $\cT z$ coordinate by coordinate is $O(mp+m^2)+m\mathfrak{M}[\{x^k,\nabla f(x^k),(i)\}\to \nabla f(\hat{x}^{k+1})]$ operations. So caching $Ax$ is still favorable unless $m\ll p$, but in extended monotropic programming we usually have $p<m$ because the linear constraint $Ax=b$ is usually under-determined.
\begin{remark}
In general, when the $s$ update is affine, we can decouple $s^{k+1}$ and $x^{k+1}$ by plugging the $s$ update into the $x$ update. It is the case when $h$ is affine or quadratic in problem \eqref{pdproblem}.
\end{remark}
A sufficient condition for $\cT_\textnormal{emp}$  to be CF is $\prox_g\in\cC_1$ i.e., separable. Indeed, we have $\cT_\textnormal{emp}=\cT_1\circ\cT_2$, where $$\cT_1=\begin{bmatrix}
\cI & 0\\0&\prox_{\eta g}
\end{bmatrix}, \cT_2\begin{bmatrix}
s\\x
\end{bmatrix}=\begin{bmatrix}
s+\gamma (Ax-b)\\
x-\eta(\nabla f(x)+A^\top s+2\gamma A^\top Ax-2\gamma A^\top b)
\end{bmatrix}.$$
  Following Case 5 of Table $\ref{table:comp-op}$, $\cT_\textnormal{emp}$ is CF. When $O(m)=O(p)$, the separability condition on $\prox_g$ can be relaxed to $\prox_g\in\cF_1$ since in this case $\cT_2\in\cF_2$, and we can apply  Case 7 of Table $\ref{table:comp-op}$ (by maintaining $\nabla f(x)$, $A^\top s$, $Ax$ and $A^\top Ax$.)
%, with $\prox_{\gamma h^*}$ possibly not in $\cC_2$ anymore. In that case we have
%\begin{equation}
%\left\{
%\begin{array}{l}
%s^{k+1}=B(s^k+\gamma Ax^k)+b\\
%x^{k+1}=\prox_{\eta g}(x^k-\eta(\nabla f(x^k)-A^\top s^k+2A^\top Bs^k+2\gamma A^\top BAx^k+2A^\top b))
%\end{array}
%\right.\label{vucondatquadratic}
%\end{equation}
%for some matrix $B$ and vector $b$. In that case $\eqref{vucondatquadratic}$ is more favorable than $\eqref{vucondat}$, because when updating $x^{k+1}_i$, instead of computing the entire $s^{k+1}$, which takes $O(p^2)$ operations, we only need to compute quantities such as $(A^\top Bs)_i$ and $(A^\top BAx)_i$, which both take $O(p)$ operations if we cache $Ax$.
%{When we have $m\ll p$. Then, instead of using the iteration \eqref{vucondat2}, we can still use \eqref{pdemp} and cache $A^\top s$ instead of $Ax$ since the former is smaller. %In that way we still have a CF operator, with the same coordinate update costs as $\cT_2$.}
%In this case, if we cache $A^\top A$ and $A^\top b$, refresh $A^\top s$ whenever a coordinate of $s$ is updated and refresh $Ax$, $A^\top A x$ and $\nabla f(x)$ whenever a coordinate of $x$ is updated, we have a BC-friendly algorithm with reduced coordinate update complexity compared to directly apply $\eqref{vucondatemp}$
%\begin{remark}
%All the above schemes have provable convergence but we neglect them due to the purpose of this paper. The efficiency of the schemes are increasing from the first one to the last one since less and less information is discarded. When $m_j=1$ for every $j$, the second and the third scheme will be the same
%\end{remark}
\subsection{Overlapping-Block Coordinate Updates}\label{sec:overlap}
{In the coordinate update scheme proposed in \S\ref{sec:pdcu}, updating $x_i$ involves computing $s^{k+1}$, because the variables   $x_i$'s and $s_j$'s are coupled through the matrix $A$. However, once $x_i^{k+1}$ is obtained, $s^{k+1}$  is discarded. It is not used to update $s$ or  cached for further use. This subsection introduces means to reduce such wasted computation. %(see the proof of Proposition \ref{prop1}, item 1). 
%Such wasteful effort is a result that  $x_i$'s and $s_j$'s are coupled through the matrix $A$.

We define, for each $i$,  $\JJ(i)\subset [p]$ as the set of indices $ j$ such that $A^\top_{i,j}\neq 0$, and, for each $j$,  $\II(j)\subset [m]$ as the set of indices of $i$ such that $A_{j,i}\neq 0$. We also let $m_j:=|\II(j)|$, and assume $m_j\neq 0,\forall j \in [p]$ without loss of generality.
%otherwise we can simply delete the $j$th column of $A$ without affecting the solution of problem $\eqref{pdproblem}$.

We arrange the coordinates of $z=[x;s]$ into $m$ (overlapping) blocks. The $i$th block consists of the coordinate $x_i$ and  $s_j$ for all $j\in\JJ(i)$. We propose a block coordinate update scheme based on $\eqref{vucondat}$. Because the blocks overlap, the $s_j$ update is relaxed with parameters $\rho_{i,j}\ge 0$ (see~\eqref{pdoverlap} below) that satisfy  $\sum_{i\in \II(j)}\rho_{i,j}=1,~~\forall j \in [p],$ so that the aggregated effect is to update $s_j$ without scaling. (Following the
%Krasnosel'ski$\breve{\text{i}}$-Mann
KM iteration~\cite{krasnosel1955two}, we can also assign a relaxation parameter $\eta_k$ for the $x_i$ update; then, the $s_j$ update should be relaxed with $\rho_{i,j}\eta_k$.)

%The new coordinate update scheme will work in the following manner: every time we choose a coordinate $x_i$ of $x$; compute the needed $s_j$'s (those with $j\in\JJ(i)$) and update them with scaling parameters $\rho_{i,j}\geq 0$; then update $x_i$ using the computed $s_j$.
We propose the following  update scheme:
\begin{equation}
\left\{
\begin{array}{l}
\text{select }i\in[m], \text{ and then compute}\\
\quad\tilde{s}_j^{k+1}=\prox_{\gamma h_j^*} (s_j^k+\gamma (Ax^k)_j),~\mbox{for all}~ j\in\JJ(i),\\
\quad\tilde{x}_i^{k+1}=\prox_{\eta g_i}(x_i^k-\eta(\nabla_i f(x^k)+\sum_{j\in\JJ(i)}A_{i,j}^\top(2\tilde{s}_j^{k+1}-s_j^k))),\\
\quad\text{update }x_i^{k+1}=x_i^k+(\tilde{x}_i^{k+1}-x_i^k), \\
\quad\text{update }s_j^{k+1}=s_j^k+\rho_{i,j}(\tilde{s}_j^{k+1}-s_j^k),~\mbox{for all}~ j\in\JJ(i).
\end{array}
\right.\label{pdoverlap}
\end{equation}
%In other words, we are grouping the coordinates into overlapping groups. Each group contains one coordinate $x_i$ of $x$ and all the $s_j$'s with $j\in\JJ(i)$. $s_j$ will be updated with every $x_i$ with $i\in\II(j)$ and we want the aggregate effect to be the same as updating $s_j$ with no scaling.
%So we need
%As for the computing cost, we have no separate $s_j$ update and no significant extra cost are added to the $x_i$ update, so the computing cost is the same, if not reduced in this scheme.
%Following the
%%Krasnosel'ski$\breve{\text{i}}$-Mann
%KM iteration\cite{krasnosel1955two}, we can also assign a relaxation parameter $\eta_k$ for the $x_i$ update, then the $s_j$ update should be relaxed by $\rho_{i,j}\eta_k$.
\begin{remark}
The use of relaxation parameters $\rho_{i,j}$ makes our scheme different from that in \cite{pesquet2014class}. %, and they are also critical to the convergence.
\end{remark}
Following the assumptions and arguments in \S$\ref{sec:pdcu}$, if we maintain $Ax$, the cost for each block coordinate update is $O(p)+\nops{x}{\nabla_i f(x)}$, which is $O(\frac{1}{m}\nops{z}{\TVC z})$. Therefore the coordinate update scheme \eqref{pdoverlap} is computationally worthy.

Typical choices of $\rho_{i,j}$ include: (1) one of the $\rho_{i,j}$'s is 1 for each $j$, others all equal to 0. This can be viewed as assigning $s_j$ fully to a block containing $x_i$.
%\begin{equation}
%\left\{
%\begin{array}{l}
%\text{when }x_i\text{ is chosen to be updated, compute}\\
%\tilde{s_j}^{k+1}=\prox_{\gamma h_j^*} (s_j^k+\gamma Ax^k),\forall j\in\JJ(i)\\
%\tilde{x_i}^{k+1}=\prox_{\eta g_i}(x_i^k-\eta(\nabla_i f(x^k)+\sum_{j\in\JJ(i)}A_{j,i}^\top(2\tilde{s}_j^{k+1}-s_j^k)))\\
%\text{update }x_i^{k+1}=x_i^k+(\tilde{x}_i^{k+1}-x_i^k) \\
%\text{update }s_j^{k+1}=s_j^k+(\tilde{s}_j^{k+1}-s_j^k) \text{ if }s_j\text{ is assigned to }x_i
%\end{array}
%\right.
%\end{equation}
(2) $\rho_{i,j}=\frac{1}{m_j}$ for all $i\in\II(j)$.
%\end{enumerate}
%\begin{equation}
%\left\{
%\begin{array}{l}
%\text{when }x_i\text{ is chosen to be updated, compute}\\
%\tilde{s_j}^{k+1}=\prox_{\gamma h_j^*} (s_j^k+\gamma Ax^k),\forall j\in\JJ(i)\\
%\tilde{x_i}^{k+1}=\prox_{\eta g_i}(x_i^k-\eta(\nabla_i f(x^k)+\sum_{j\in\JJ(i)}A_{j,i}^\top(2\tilde{s}_j^{k+1}-s_j^k)))\\
%\text{update }x_i^{k+1}=x_i^k+(\tilde{x}_i^{k+1}-x_i^k) \\
%\text{update }s_j^{k+1}=s_j^k+\frac{1}{m_j}(\tilde{s}_j^{k+1}-s_j^k),\forall j\in\JJ(i)
%\end{array}
%\right.\label{pd3rd}
%\end{equation}

\begin{remark}
The recent paper \cite{fercoq2015coordinate} proposes a different primal-dual coordinate update algorithm.
The authors produce a new matrix $\bar{A}$ based on $A$, with only one nonzero entry in each row, i.e. $m_j=1$ for each $j$. They also modify $h$ to $\bar{h}$ so that the problem
\begin{equation}
\underset{x\in\mathbb{H}}{\textnormal{minimize }} f(x)+g(x)+\bar{h}(\bar{A}x)\label{fbpdproblem}
\end{equation}
has the same solution as $\eqref{pdproblem}$. Then they solve $\eqref{fbpdproblem}$ by the scheme $\eqref{pdoverlap}$. Because they have $m_j=1$, every dual variable coordinate is only associated with one primal variable coordinate.
They create non-overlapping blocks of $z$ by duplicating each dual variable coordinate $s_j$ multiple times. The computation cost for each block coordinate update of their algorithm is the same as $\eqref{pdoverlap}$, but  more memory is needed for the duplicated copies of each $s_j$.
\end{remark}


\subsection{Async-Parallel Primal-Dual Coordinate Update Algorithms and Their Convergence}

In this subsection, we propose two async-parallel primal-dual coordinate update algorithms using the algorithmic framework of~\cite{Peng_2015_AROCK} and state their convergence results.
When there is only one agent, all algorithms proposed in this section reduce to stochastic coordinate update algorithms~\cite{Patrick_2015}, and their convergence is a direct consequence of Theorem~\ref{thm:async-convergence}. Moreover, our convergence analysis also applies to sync-parallel algorithms.



The two algorithms are based on~\S\ref{sec:pdcu} and~\S\ref{sec:overlap}, respectively.


%We assume a shared memory architecture and allow delays and inconsistent reading (see below for explanations).



\begin{algorithm}[H]\label{alg:asyn_core}
%\SetAlgoLined
\SetKwInOut{Input}{Input}\SetKwInOut{Output}{output}
\Input{$z^0\in\FF$,  $K>0$, a discrete distribution $(q_1,\ldots,q_{m+p})$
with $\sum_{i=1}^{m+p}q_i=1$ and $q_i>0,\forall i$, } set global iteration
counter $k=0$\; \While{$k < K$, every agent asynchronously and continuously}{
  select $i_k\in [m+p]$ with $\mathrm{Prob}(i_k=i)=q_i$\;%from $\{1,
  % ..., m\}$ under a discrete positive distribution\; $\hat{x}^k\gets$ reads the variable $x$\;
  % computes $\Delta x=-\frac{\eta_k}{mp_{i_k}} S_{i_k}\hat{x}^k$; ~(see \eqref{eqn:asyn_update}--\eqref{eqn:inconsist} for definitions)\;
  % adds $(\Delta x)_{i_k}$ to the component $x_{i_k}$ of the variable $x$\;
  perform an update to $z_{i_k}$ according to  \eqref{eqn:asyn_update}\;
  update the global counter $k \leftarrow k+1$\;
 }
 \caption{Async-parallel primal-dual coordinate update algorithm using $\TVC$}
\end{algorithm}

Whenever an agent updates a coordinate, the global iteration number $k$ increases by one.
The $k$th update is applied to $z_{i_k}$, with $i_k$ being independent random variables: $z_i=x_i$ when $i\leq m$ and $z_i=s_{i-m}$ when $i>m$.  Each coordinate update has
the form:
\begin{equation}
\label{eqn:asyn_update}
\left\{
\begin{array}{l}
z_{i_k}^{k+1} = z_{i_k}^k - \frac{\eta_k}{(m+p)q_{i_k}} \, (\hat{z}^k_{i_k}-(\TVC \hat{z}^k)_{i_k}),\\
z_i^{k+1}=z_i^k,\quad \forall i\neq i_k,
\end{array}
\right.
\end{equation}
where $\eta_k$ is the step size, $z^k$ denotes the state of $z$ in global memory just before the update~\eqref{eqn:asyn_update} is applied, and $\hat{z}^k$ is the result that $z$ in global memory is read by an agent to its local cache. While $(\hat{z}^k_{i_k}-(\TVC \hat{z}^k)_{i_k})$ is being computed, asynchronous parallel computing allows other agents to make updates to $z$, introducing so-called asynchronous delays. Therefore, $\hat{z}^k$
can be different from $z^k$. We refer the reader to~\cite[\S 1.2]{Peng_2015_AROCK} for more details.

The async-parallel algorithm using the overlapping-block coordinate update~\eqref{pdoverlap} is in Algorithm~\ref{alg:asyn_overlap} (recall the overlapping-block coordinate update is introduced to save computation).

\begin{algorithm}[H]\label{alg:asyn_overlap}
%\SetAlgoLined
\SetKwInOut{Input}{Input}\SetKwInOut{Output}{output}
\Input{$z^0\in\FF$,  $K>0$, a discrete distribution $(q_1,\ldots,q_{m})$
with $\sum_{i=1}^{m}q_i=1$ and $q_i>0,\forall i$, } set global iteration
counter $k=0$\; \While{$k < K$, every agent asynchronously and continuously}{
  select $i_k\in [m]$ with $\mathrm{Prob}(i_k=i)=q_i$\;%from $\{1,
  % ..., m\}$ under a discrete positive distribution\; $\hat{x}^k\gets$ reads the variable $x$\;
  % computes $\Delta x=-\frac{\eta_k}{mp_{i_k}} S_{i_k}\hat{x}^k$; ~(see \eqref{eqn:asyn_update}--\eqref{eqn:inconsist} for definitions)\;
  % adds $(\Delta x)_{i_k}$ to the component $x_{i_k}$ of the variable $x$\;
  Compute $\tilde{s}_j^{k+1},\forall j\in\JJ(i_k)$ and  $\tilde{x}_{i_k}^{k+1}$ according to $\eqref{pdoverlap}$\;
  update $x_i^{k+1}=x_i^k+\frac{\eta_k}{mq_{i_k}}(\tilde{x}_i^{k+1}-x_i^k)$\;
  update $s_j^{k+1}=s_j^k+\frac{\rho_{i,j}\eta_k}{mq_{i_k}}(\tilde{s}_j^{k+1}-s_j^k),~\mbox{for all}~ j\in\JJ(i_k)$\;
  update the global counter $k \leftarrow k+1$\;
 }
 \caption{Async-parallel primal-dual overlapping-block coordinate update algorithm using $\TVC$}
\end{algorithm}
Here we still allow asynchronous delays, so $\tilde{x}_{i_k}$ and $\tilde{s}_j^{k+1}$ are computed using some $\hat{z}^k$.
\begin{remark}
If shared memory is used, it is recommended to set all but one $\rho_{i,j}$'s to $0$ for each $i$.
\end{remark}

%Now we state the convergence theorem for Algorithm~\ref{alg:asyn_core} and~\ref{alg:asyn_overlap}.
\begin{thm}\label{thm:async-convergence}
Let $Z^*$ be the set of  solutions to problem~\eqref{pdproblem} and $(z^k)_{k\geq0}\subset \FF$ be the sequence generated by Algorithm \ref{alg:asyn_core} or Algorithm \ref{alg:asyn_overlap} under the following conditions:
\begin{enumerate}[(i)]
\item $f,g,h^*$ are closed proper convex functions, $f$ is differentiable, and $\nabla f$ is Lipschitz continuous with constant $\beta$;
\item the delay for every coordinate is bounded by a positive number $\tau$, i.e. for every $1\leq i\leq m+p$, $\hat{z}^{k}_{i}=z_i^{k-d_{i}^k}$ for some $0\leq d_{i}^k\leq\tau$;
\item $\eta_k
\in [\eta_{\min}, \eta_{\max}]$ for certain $\eta_{\min},\eta_{\max}>0$.
\end{enumerate}
Then $(z^k)_{k\geq 0}$ converges to a $Z^*$-valued random variable with probability 1.
\end{thm}
The  formulas for $\eta_{\min}$ and $\eta_{\max}$, as well as the proof of Theorem~\ref{thm:async-convergence},  are given in Appendix \ref{pf:pdasync} along with additional remarks. The algorithms can be applied to solve  problem~\eqref{pdproblem}. A variety of examples are provided in~\S\ref{sec:ML} and~\S\ref{sec:Imaging}.

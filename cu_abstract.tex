This paper focuses on the \emph{coordinate update method}, which can be used to large-sized problems involving linear and nonlinear mappings, and smooth and nonsmooth functions. It decomposes a problem into simple subproblems, where each subproblem updates one, or a small block of, variables. The coordinate update method sits at a high level of abstraction and includes many special cases such as the Jacobi, Gauss-Seidel, alternated projection, as well as \emph{coordinate descent} methods. They have found greatly many applications throughout computational sciences.

In this paper, we abstract many problems to the fixed-point problem $x=\mathcal{T} x$ and study the favorable structures in the operator $\mathcal{T}$ such that the coordinate update $x_i^{k+1} = (\mathcal{T} x^k)_i$ can be performed at a relatively low cost. The selection of updating index can be sequential, parallel, or async-parallel. 

This study leads to new coordinate update algorithms for a variety of problems in machine learning, image processing, as well as sub-areas of optimization. The obtained algorithms are scalable to very large instances through parallel and even asynchronous computing. We present numerical examples to illustrate how effective these algorithms are.


% !TEX root = ./amsa_main.tex
\subsubsection{DRS for Image Processing in the Primal-dual Form \cite{o2014primal}}
Many convex image processing problems have the general form
$$\Min_{x} f(x) + g(Ax),$$
where $A$ is a matrix such as a dictionary, sampling operator, or finite difference operator. We can reduce the problem to the system: $0\in\cA(z) + \cB(z)$, where $z=[x;s]$,
$$\cA(z):=\begin{bmatrix}\partial f(x)\\\partial g^*(s)\end{bmatrix},\quad\mbox{and}\quad \cB(z):=\begin{bmatrix}0 & ~A^\top  \\ -A & 0\end{bmatrix}\begin{bmatrix}x\\s\end{bmatrix}.$$
(See Appendix~\ref{sec:vc-op} for the reduction.) The work \mbox{%DIFAUXCMD
\cite{o2014primal} }%DIFAUXCMD
gives their resolvents    
\begin{align*}\cJ_{\gamma \cA} &= \begin{bmatrix}\prox_{\gamma f}\\\prox_{\gamma g^*}\end{bmatrix},\\ 
\cJ_{\gamma \cB} &= (I+\gamma \cB)^{-1}= \begin{bmatrix} 0 &~ 0 \\ 0 &~ I\end{bmatrix}+\begin{bmatrix}I\\ \gamma A\end{bmatrix}(I+\gamma^2 A^\top A)^{-1}\begin{bmatrix}I\\ -\gamma A\end{bmatrix}^\top ,
\end{align*}
where $\cJ_{\gamma \cA}$ is often cheap or separable and we can \emph{explicitly form}  $\cJ_{\gamma \cB}$  as a matrix or implement it based on a fast transform. With the defined $\cJ_{\gamma \cA}$ and $\cJ_{\gamma \cB}$, we can apply the RPRS method as $z^{k+1} = \cT_{\text{RPRS}} z^k$. The resulting RPRS operator is CF when $\cJ_{\gamma \cB}$ is CF. Hence, we can derive a new RPRS coordinate update algorithm. We leave the derivation to the readers. Derivations of coordinate update algorithms for more specific image processing problems are shown in the following subsections.

\subsubsection{Total Variation Image Processing}
We consider the following Total Variation (TV) image processing model
\begin{equation}\label{eqn:tvl2}
\Min_{x}~\lambda \|x\|_{\text{TV}} + \frac{1}{2} \|A\,x - b\|^2,
\end{equation}
where $x\in \RR^n$ is the vector representation of the unknown image, $A$ is an $m \times n$ matrix describing the transformation from the image to the measurements $b$. Common $A$ includes sampling matrices in MRI, CT, denoising, deblurring, etc.  Let $(\nabla_i^h,\nabla_i^v)$ be the discrete gradient at pixel $i$. Then the TV semi-norm $\|\cdot\|_{\text{TV}}$ in the isotropic and anisotropic fashions are, respectively,
\begin{subequations}\label{eqn:tv_def}
\begin{align}
\|x\|_{\text{TV}} &= {\sum_{i} \sqrt{(\nabla_i^h x)^2 + (\nabla_i^v x)^2},}\\
  \| x\|_{\text{TV}} &= \| \nabla x\|_1 = \sum_{i} \left(|\nabla_i^h x| + |\nabla_i^v x|\right).
	\end{align}
\end{subequations}
%DIF >  or in the anisotropic fashion
%DIF >  \begin{equation}
%DIF >  \|x\|_{\text{TV}} = \|\nabla x\|_1 = \sum_{i} \left(|\nabla_i^h x| + |\nabla_i^v x|\right).
%DIF >  \end{equation}

%
%where the $\|\cdot\|_{\text{TV}}$ can be the isotropic TV norm
%\begin{equation}
%\|x\|_{\text{TV}} = \sum_{i} \sqrt{(\nabla_i^h x)^2 + (\nabla_i^v x)^2},
%\end{equation}
%where $\nabla_i^h$ and $\nabla_i^v$ are linear operators corresponding to horizontal and vertical first order differences at pixel $i$, respectively. $\|\cdot\|_{\text{TV}}$ can also be the anisotropic TV norm
%\begin{equation}
%\|x\|_{\text{TV}} = \|\nabla x\|_1 = \sum_{i} \left(|\nabla_i^h x| + |\nabla_i^v x|\right).
%\end{equation}
%
%
%Let $x \in \RR^n$ be the vector representation of the unknown image and $b \in \RR^m$ be the given measurements. Here $A$ is a $m \times n$ matrix describing the direct transformation from the image to the measurements. 
%{Consider applications in which $A$ is a sparse matrix. For example, in CT image reconstruction, $A$ models the discrete Radon transform. Each row describes a line integral, and as each line  only intersects with  a few pixels, there are many zeros in each row. For image deblurring, $A$ models a convolution kernel, and each row is sparse if the size of the kernel is small. 
%$A$ depends on the imaging modality used; for example, in Computed Tomography (CT), $A$ is the discrete Radon transform, with each row describing an integral along one straight line, and all the elements in $A$ are nonnegative. 

For simplicity, we use the anisotropic TV for analysis and in the numerical experiment in \S~\ref{sec:tv}. It is {slightly more complicated for the isotropic TV. Introducing the following notation
$$B \DIFaddbegin \DIFadd{:}\DIFaddend = \begin{pmatrix} \nabla \\ A \end{pmatrix}, \quad h (p, q) \DIFaddbegin \DIFadd{:}\DIFaddend = \lambda \|p\|_1 + \frac{1}{2} \|q - b\|^2, $$
 {we can reformulate }\eqref{eqn:tvl2} \DIFaddend as
$$\Min_x ~ h(B\,x) = h(\nabla x, A \, x),$$
which {reduces to the form of~\eqref{pdproblem} with $f=g=0$. Based on its definition, the convex conjugate of $h(p, q)$ and its proximal operator are, respectively, 
\begin{align}
h^* (s, t) &= \iota_{\|\cdot\|_{\infty} \leq \lambda} (s) + \frac{1}{2} \|t + b\|^2 - \frac{1}{2} \|b\|^2, \label{eqn:dual-h}\\
\prox_{\gamma h^*} (s, t) &= \prj_{\|\cdot\|_{\infty} \leq \lambda} (s) + \frac{1}{1 + \gamma} (t - \gamma b). \label{eqn:prox-dual-h}
\end{align}
Let $s, t$ be the dual variables corresponding to $\nabla x$ and $Ax$ respectively, then using \eqref{eqn:prox-dual-h} and applying \eqref{vucondat2} give the following full update:
\begin{subequations}\label{eqn:pd_tvl2}
\begin{align}
x^{k + 1} &= x^k - \eta (\nabla^\top s^k + A^\top  t^k), \\
s^{k + 1} &= \prj_{\|\cdot\|_{\infty} \leq \lambda} \left(s^k + \gamma \nabla (x^k - 2\eta (\nabla^\top s^k + A^\top t^k))\right), \\ 
t^{k+1} &= \frac{1}{1 + \gamma} \left(t^k + \gamma A (x^k - 2 \eta (\nabla^\top s^k + A^\top t^k)) - \gamma b \right).
\end{align}
\end{subequations}
To perform the coordinate updates {as described in \S\ref{sec:p-d}}, we can maintain $\nabla^\top s^k$ and $A^\top t^k$. Whenever a coordinate of $(s,t)$ is updated, the corresponding $\nabla^\top s^k$ (or $A^\top t^k)$ should also be updated. Specifically, we have the following coordinate update algorithm
\begin{equation}\label{update-lap-gl}
{\left\{
\begin{array}{l}
\text{if $x_i$  is chosen for some $i \in [n]$, then compute}\\
\qquad x_i^{k + 1} = x_i^k - \eta (\nabla^\top s^k + A^\top  t^k)_i; \\
\text{if $s_i$  is chosen for some $i \in [2n]$, then compute}\\
\qquad s_i^{k + 1} = \prj_{\|\cdot\|_{\infty} \leq \lambda} \left(s_i^k + \gamma \nabla_i (x^k - 2\eta (\nabla^\top s^k + A^\top t^k))\right) \\
\qquad \text{and update $\nabla^\top s^k$ to $\nabla^\top s^{k+1}$}; \\
\text{if $t_i$  is chosen for some $i \in [m]$, then compute}\\
\qquad t_i^{k+1} = \frac{1}{1 + \gamma} \left(t_i^k + \gamma A_{i,:} (x^k - 2 \eta (\nabla^\top s^k + A^\top t^k)) - \gamma b_i \right)\\
\qquad \text{and update $A^{\top} t^k$ to $A^{\top} t^{k+1}$}. \\
\end{array}
\right.
}\end{equation}




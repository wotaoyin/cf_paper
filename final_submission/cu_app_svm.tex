% !TEX root = ./amsa_main.tex
\subsubsection{Support Vector Machine}\label{sec:svm}
Given the training data $\{(a_i,\beta_i)\}_{i=1}^m$ with $\beta_i\in\{+1, -1\},\,\forall i$, the kernel support vector machine \cite{scholkopf2001learning} is 
\begin{equation}\label{eq:ksvm}
\begin{array}{rl}
\Min\limits_{x,\xi,y} &\frac{1}{2}\|x\|_2^2+C\sum_{i=1}^m \xi_i,\\
 \st\ & \beta_i(x^\top\phi(a_i)-y)\ge 1-\xi_i,\, \xi_i\ge 0,\, \forall i \in [m],
\end{array}
\end{equation}
where $\phi$ is a vector-to-vector map, mapping each data $a_i$ to a point in a (possibly) higher-dimensional space. If $\phi(a)=a$, then \eqref{eq:ksvm} reduces to the linear support vector machine. The model~\eqref{eq:ksvm}  can be interpreted as finding a hyperplane $\{w:x^\top w-y=0\}$ to separate two sets of points $\{\phi(a_i):\beta_i=1\}$ and $\{\phi(a_i):\beta_i=-1\}$. 

The dual problem of \eqref{eq:ksvm} is
\begin{equation}\label{eq:dksvm}
\Min_s \frac{1}{2}s^\top Q s- e^\top s,\ \St\ 0\le s_i\le C,\,\forall i,\, \sum_i \beta_i s_i=0,
\end{equation}
where $Q_{ij}=\beta_i\beta_j k(a_i,a_j)$, $k(\cdot,\cdot)$ is a so-called \emph{kernel function}, and $e = (1, ..., 1)^\top$. If $\phi(a)=a$, then $k(a_i,a_j)=a_i^\top a_j$.

\subsubsection*{Unbiased case}
If $y=0$ is enforced in \eqref{eq:ksvm}, then the solution hyperplane $\{w:x^\top w=0\}$ passes through the origin and is called \emph{unbiased}. Consequently, the dual problem \eqref{eq:dksvm} will no longer have the linear constraint $\sum_i \beta_i s_i=0$, leaving it with the coordinate-wise separable box constraints $0\le s_i\le C$.  To solve \eqref{eq:dksvm}, we can apply the FBS operator $\cT$ defined by~\eqref{eq:FBS}. Let $d(s):= \frac{1}{2}s^\top Q s- e^\top s$}, $\cA = \prox_{[0, C]}$, and $\cC = \nabla d$.  The coordinate update based on FBS is
$$s_i^{k+1}=\prj_{[0,C]}(s_i^k-\gamma_i\nabla_i d(s^k)),$$
where we can take $\gamma_i=\frac{1}{Q_{ii}}$. 

\subsubsection*{Biased (general) case} In this case, the mode \eqref{eq:ksvm} has $y\in\RR$, so the hyperplane $\{w:x^\top w-y=0\}$ may not pass the origin and is called \emph{biased}. Then,  the dual problem \eqref{eq:dksvm} retains the linear constraint $\sum_i \beta_i s_i=0$. In this case, we apply the primal-dual splitting scheme \eqref{vucondat} or the three-operator splitting scheme \eqref{3s}.

The coordinate update based on the full primal-dual splitting scheme~\eqref{vucondat} is:
\begin{subequations}\label{eq:pd-svm}
\begin{align}
t^{k+1}=&\,t^k+\gamma\sum_{i=1}^m \beta_i s_i^k,\label{eq:pd-svm-t}\\
s_i^{k+1}=&\,\prj_{[0,C]}\left(s_i^k-\eta\big(\nabla_i d(s^k)+\beta_i(2t^{k+1}-t^k)\big)\right), \label{eq:pd-svm-s}
\end{align}
\end{subequations}
{where $t,s$ are the primal and dual variables, respectively. Note that we can let $w:=\sum_{i=1}^m \beta_i s_i$ and maintain it. With variable $w$ and substituting \eqref{eq:pd-svm-t} into \eqref{eq:pd-svm-s}, we can equivalently write \eqref{eq:pd-svm} into

\begin{equation}
{\left\{
\begin{array}{l}
\text{if }t\text{ is chosen (the index 0), then compute}\\
\qquad t^{k+1}=\,t^k+\gamma w^k,\\
\text{if }s_i\text{ is chosen (an index $i \in [m]$), then compute}\\
\qquad s_i^{k+1}=\,\prj_{[0,C]}\left(s_i^k-\eta\big(q_i^\top s^k-1+\beta_i(2\gamma w^k+t^k)\big)\right)\\
\qquad w^{k+1}= w^k + \beta_i (s_i^{k+1} - s_i^k).
\end{array}
\right.
}\end{equation}


%\begin{subequations}\label{eq:pd-svm2}
%\begin{align}
%t^{k+1}=&\,t^k+\gamma w^k,\label{eq:pd-svm2-t}\\
%s_i^{k+1}=&\,\prj_{[0,C]}\left(s_i^k-\eta\big(q_i^\top s^k-1+\beta_i(2\gamma w^k+t^k)\big)\right),\,i=1,\ldots,m.\label{eq:pd-svm2-s}
%\end{align}
%\end{subequations}
%In parallel computing, whenever a processor updates some $s_i$, the $w$ variable must be also renewed as $w^{k+1}=w^k+\beta_i(s_i^{k+1}-s_i^k)$.

We can also apply the three-operator splitting  \eqref{3s} as follows. Let {$D_1:=[0,C]^m$ and $D_2:=\{s: \sum_{i=1}^m \beta_i s_i=0\}$. Let $\cA = \prj_{D_2}$, $\cB = \prj_{D_1}$, and $\cC(x) = Qx - e$,  The full update corresponding to $\cT = (I - \eta_k) \cI + \eta_k \TS$ is
\begin{subequations}\label{eq:3op-svm}
\begin{align}
s^{k+1}=&\prj_{D_2}(u^k),\label{eq:3op-svm-s}\\
u^{k+1}=&u^k+\eta_k\left(\prj_{D_1}\big(2s^{k+1}-u^k-\gamma (Qs^{k+1}-e)\big)-s^{k+1}\right)\label{eq:3op-svm-u},
\end{align}
\end{subequations}
where $s$ is just an intermediate variable.
Let $\tilde{\beta}:=\frac{\beta}{\|\beta\|_2}$ and $w:=\tilde{\beta}^\top u$. Then $\prj_{D_2}(u)=(I-\tilde{\beta}\tilde{\beta}^\top)u$. Hence, $s^{k+1}=u^k-w^k\tilde{\beta}$. Plugging it into~\eqref{eq:3op-svm-u} yields the following coordinate update scheme:
\begin{equation*}
{\left\{
\begin{array}{l}
\text{if }i \in [m]\text{ is chosen, then compute}\\
\quad s^{k+1}_i=u^k_i-w^k\tilde{\beta}_i,\\
\quad u^{k+1}_i=u^k_i+\eta_k\left(\prj_{[0,C]}\left(2s_i^{k+1}-u^k_i-\gamma \big(q_i^\top u^k- w^k (q_i^\top \tilde{\beta})-1\big)\right)-s_i^{k+1}\right) \\
\quad w^{k+1}=w^k+\tilde{\beta}_i(u_i^{k+1}-u_i^k),
\end{array}
\right.
}\end{equation*}
where $w^k$ is the maintained variable and $s^k$ is the intermediate variable.


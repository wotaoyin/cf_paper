%

\documentclass[amsa]{ipart}

\newif\ifamsa % declare the amsa switch
\amsatrue     % set the switch to true

\RequirePackage[OT1]{fontenc}
\RequirePackage{amsthm,amsmath,amssymb}
\RequirePackage[numbers,square]{natbib} %
\RequirePackage[colorlinks]{hyperref}

\newtheorem{thm}{Theorem}


%% macros for commenting
\usepackage{mathtools}
\usepackage[normalem]{ulem} % to use \sout
%\usepackage{algorithm,algorithmic}
%\usepackage{graphicx, subfigure}
\usepackage{subcaption}
%\usepackage{url}
\usepackage{enumerate}
\usepackage[ruled,vlined]{algorithm2e}
\usepackage{amsfonts}
\usepackage{amssymb}
\usepackage{booktabs}
\usepackage{multirow}


%% a light-weight algorithm environment
\newtheorem{algo}{Algorithm}

%% highlighting and commenting
\newcommand{\outline}[1]{{\color{brown}#1}}
\newcommand{\rev}[1]{{\color{blue}#1}}
\newcommand{\commwy}[1]{{\color{red}(#1 -- Wotao)}} % Wotao Yin's comments
\newcommand{\commyx}[1]{{\color{red}(#1 -- Yangyang)}} % Yangyang Xu's comments
\newcommand{\commzp}[1]{{\color{red}(#1 -- Zhimin)}} % Yangyang Xu's comments
\newcommand{\commtw}[1]{{\color{red}(#1 -- Tianyu)}} 
\newcommand{\commrw}[1]{{\color{red}(#1 -- Reviewer)}} % Yangyang Xu's comments
\newcommand{\remove}[1]{{}}
\newcommand{\cut}[1]{}

%% macros for letters

\newcommand{\va}{{\mathbf{a}}}
\newcommand{\vb}{{\mathbf{b}}}
\newcommand{\vc}{{\mathbf{c}}}
\newcommand{\vd}{{\mathbf{d}}}
\newcommand{\ve}{{\mathbf{e}}}
\newcommand{\vf}{{\mathbf{f}}}
\newcommand{\vg}{{\mathbf{g}}}
\newcommand{\vh}{{\mathbf{h}}}
\newcommand{\vi}{{\mathbf{i}}}
\newcommand{\vj}{{\mathbf{j}}}
\newcommand{\vk}{{\mathbf{k}}}
\newcommand{\vl}{{\mathbf{l}}}
\newcommand{\vm}{{\mathbf{m}}}
\newcommand{\vn}{{\mathbf{n}}}
\newcommand{\vo}{{\mathbf{o}}}
\newcommand{\vp}{{\mathbf{p}}}
\newcommand{\vq}{{\mathbf{q}}}
\newcommand{\vr}{{\mathbf{r}}}
\newcommand{\vs}{{\mathbf{s}}}
\newcommand{\vt}{{\mathbf{t}}}
\newcommand{\vu}{{\mathbf{u}}}
\newcommand{\vv}{{\mathbf{v}}}
\newcommand{\vw}{{\mathbf{w}}}
\newcommand{\vx}{{\mathbf{x}}}
\newcommand{\vy}{{\mathbf{y}}}
\newcommand{\vz}{{\mathbf{z}}}
%
%\newcommand{\ta}{{\tilde{a}}}
%\newcommand{\tb}{{\tilde{b}}}
%\newcommand{\tc}{{\tilde{c}}}
%\newcommand{\td}{{\tilde{d}}}
%\newcommand{\te}{{\tilde{e}}}
%\newcommand{\tf}{{\tilde{f}}}
%\newcommand{\tg}{{\tilde{g}}}
%\newcommand{\th}{{\tilde{h}}}
%\newcommand{\ti}{{\tilde{i}}}
%\newcommand{\tj}{{\tilde{j}}}
%\newcommand{\tk}{{\tilde{k}}}
%\newcommand{\tl}{{\tilde{l}}}
%\newcommand{\tm}{{\tilde{m}}}
%\newcommand{\tn}{{\tilde{n}}}
%\newcommand{\to}{{\tilde{o}}}
%\newcommand{\tp}{{\tilde{p}}}
%\newcommand{\tq}{{\tilde{q}}}
%\newcommand{\tr}{{\tilde{r}}}
%\newcommand{\ts}{{\tilde{s}}}
%\newcommand{\tt}{{\tilde{t}}}
%\newcommand{\tu}{{\tilde{u}}}
\newcommand{\tv}{{\tilde{v}}}
\newcommand{\tw}{{\tilde{w}}}
%\newcommand{\tx}{{\tilde{x}}}
%\newcommand{\ty}{{\tilde{y}}}
\newcommand{\tz}{{\tilde{z}}}
\newcommand{\umu}{\overline{M}}
\newcommand{\lmu}{\underline{M}}

\newcommand{\vA}{{\mathbf{A}}}
\newcommand{\vB}{{\mathbf{B}}}
\newcommand{\vC}{{\mathbf{C}}}
\newcommand{\vD}{{\mathbf{D}}}
\newcommand{\vE}{{\mathbf{E}}}
\newcommand{\vF}{{\mathbf{F}}}
\newcommand{\vG}{{\mathbf{G}}}
\newcommand{\vH}{{\mathbf{H}}}
\newcommand{\vI}{{\mathbf{I}}}
\newcommand{\vJ}{{\mathbf{J}}}
\newcommand{\vK}{{\mathbf{K}}}
\newcommand{\vL}{{\mathbf{L}}}
\newcommand{\vM}{{\mathbf{M}}}
\newcommand{\vN}{{\mathbf{N}}}
\newcommand{\vO}{{\mathbf{O}}}
\newcommand{\vP}{{\mathbf{P}}}
\newcommand{\vQ}{{\mathbf{Q}}}
\newcommand{\vR}{{\mathbf{R}}}
\newcommand{\vS}{{\mathbf{S}}}
\newcommand{\vT}{{\mathbf{T}}}
\newcommand{\vU}{{\mathbf{U}}}
\newcommand{\vV}{{\mathbf{V}}}
\newcommand{\vW}{{\mathbf{W}}}
\newcommand{\vX}{{\mathbf{X}}}
\newcommand{\vY}{{\mathbf{Y}}}
\newcommand{\vZ}{{\mathbf{Z}}}

\newcommand{\cA}{{\mathcal{A}}}
\newcommand{\cB}{{\mathcal{B}}}
\newcommand{\cC}{{\mathcal{C}}}
\newcommand{\cD}{{\mathcal{D}}}
\newcommand{\cE}{{\mathcal{E}}}
\newcommand{\cF}{{\mathcal{F}}}
\newcommand{\cG}{{\mathcal{G}}}
\newcommand{\cH}{{\mathcal{H}}}
\newcommand{\cI}{{\mathcal{I}}}
\newcommand{\cJ}{{\mathcal{J}}}
\newcommand{\cK}{{\mathcal{K}}}
\newcommand{\cL}{{\mathcal{L}}}
\newcommand{\cM}{{\mathcal{M}}}
\newcommand{\cN}{{\mathcal{N}}}
\newcommand{\cO}{{\mathcal{O}}}
\newcommand{\cP}{{\mathcal{P}}}
\newcommand{\cQ}{{\mathcal{Q}}}
\newcommand{\cR}{{\mathcal{R}}}
\newcommand{\cS}{{\mathcal{S}}}
\newcommand{\cT}{{\mathcal{T}}}
\newcommand{\cU}{{\mathcal{U}}}
\newcommand{\cV}{{\mathcal{V}}}
\newcommand{\cW}{{\mathcal{W}}}
\newcommand{\cX}{{\mathcal{X}}}
\newcommand{\cY}{{\mathcal{Y}}}
\newcommand{\cZ}{{\mathcal{Z}}}

\newcommand{\ri}{{\mathrm{i}}}
\newcommand{\rr}{{\mathrm{r}}}


%% macros for math notions and operators

\newcommand{\RR}{\mathbb{R}}
\newcommand{\NN}{\mathbb{N}}
\newcommand{\CC}{\mathbb{C}}
\newcommand{\HH}{\mathbb{H}}
\newcommand{\II}{\mathbb{I}}
\newcommand{\JJ}{\mathbb{J}}
\newcommand{\DD}{\mathbb{D}}
\newcommand{\GG}{\mathbb{G}}
\newcommand{\ZZ}{\mathbb{Z}}
\renewcommand{\SS}{{\mathbb{S}}}
\newcommand{\SSp}{\mathbb{S}_{+}}
\newcommand{\SSpp}{\mathbb{S}_{++}}
\newcommand{\sign}{\mathrm{sign}}
\newcommand{\vzero}{\mathbf{0}}
\newcommand{\vone}{{\mathbf{1}}}

%%Probability symbols.
\newcommand{\EE}{\mathbb{E}}
\newcommand{\mkT}{\mathfrak{T}}
\newcommand{\mkS}{\mathfrak{S}}
\newcommand{\mkQ}{\mathfrak{Q}}
\newcommand{\pr}{\mathrm{prod}}

\newcommand{\st}{{\text{s.t.}}} % subject to
\newcommand{\St}{{\mathrm{subject~to}}} % subject to
\newcommand{\op}{{\mathrm{op}}} % subscript for operator norm
\newcommand{\opt}{{\mathrm{opt}}} % subscript for optimal solution
%\newcommand{\supp}{{\mathrm{supp}}} % support
\newcommand{\Prob}{{\mathrm{Prob}}} % probability
\newcommand{\Diag}{{\mathrm{Diag}}} % vector -> diagonal matrix
%\newcommand{\diag}{{\mathrm{diag}}} % matrix diagonal -> vector
\newcommand{\dom}{{\mathrm{dom}}} % domain
\newcommand{\range}{{\mathrm{range}}} % domain
%\newcommand{\grad}{{\nabla}}    % gradient
\newcommand{\tr}{{\mathrm{tr}}} % trace
\newcommand{\TV}{{\mathrm{TV}}} % total variation
\newcommand{\Proj}{{\mathrm{Proj}}}
\newcommand{\prj}{{\mathbf{proj}}}
\newcommand{\prox}{\mathbf{prox}}
\newcommand{\reflh}{\refl^{\bH}}
\newcommand{\proxh}{\prox^{\bH}}
\newcommand{\minimize}{\text{minimize}}
\newcommand{\bgamma}{\boldsymbol{\gamma}}
\newcommand{\bsigma}{\boldsymbol{\sigma}}
\newcommand{\bomega}{\boldsymbol{\omega}}
\newcommand{\blambda}{\boldsymbol{\lambda}}
\newcommand{\bH}{\vH}
\newcommand{\bbH}{\mathbb{H}}
\newcommand{\bB}{\boldsymbol{\cB}}
\newcommand{\Tau}{\mathrm{T}}
\newcommand{\tnabla}{\widetilde{\nabla}}
\newcommand{\TS}{{\cT_{\mathrm{3S}}}}
\newcommand{\TFBS}{{\cT_{\mathrm{FBS}}}}
\newcommand{\TBFS}{{\cT_{\mathrm{BFS}}}}
\newcommand{\TDRS}{{\cT_{\mathrm{DRS}}}}
\newcommand{\TPRS}{{\cT_{\mathrm{PRS}}}}
\newcommand{\TFBFS}{{\cT_{\mathrm{FBFS}}}}
\newcommand{\TFDRS}{{\cT_{\mathrm{FDRS}}}}
\newcommand{\TVC}{{\cT_{\textnormal{CV}}}}
\newcommand{\best}{\mathrm{best}}
\newcommand{\kbest}{k_{\best}}
\newcommand{\diff}{\mathrm{diff}}
\newcommand{\xbar}{\bar{x}}
\newcommand{\xgbar}{\bar{x}_g}
\newcommand{\xfbar}{\bar{x}_f}
\newcommand{\xihat}{\hat{\xi}}
\newcommand{\xg}{x_g}
\newcommand{\xf}{x_f}
\newcommand{\du}{\mathrm{d}u}
\newcommand{\dy}{\mathrm{d}y}
\newcommand{\kconvergence}{\stackrel{k \rightarrow \infty}{\rightarrow }}
\newcommand{\Grph}{\mathrm{Grph}}
\DeclareMathOperator{\shrink}{shrink} % shrinkage
\DeclareMathOperator*{\argmin}{arg\,min}
\DeclareMathOperator*{\argmax}{arg\,max}
\DeclareMathOperator*{\Min}{minimize}
\DeclareMathOperator*{\Max}{maximize}
\DeclareMathOperator*{\Fix}{Fix}
\DeclareMathOperator*{\zer}{zer}    % the set of zeros of an operator
\DeclareMathOperator*{\nablah}{\nabla^{\bH}}
\DeclareMathOperator*{\gra}{gra}
\DeclarePairedDelimiter{\dotpb}{\langle}{\rangle_{\bH}}
\DeclarePairedDelimiter{\dotpv}{\langle}{\rangle_{\vH}}
\DeclareMathOperator*{\as}{a.s.}
\newcommand{\nops}[2]{\ensuremath{\mathfrak{M}\left[{#1}\mapsto {#2}\right]}}


%% macros for environments math equations

\newcommand{\MyFigure}[1]{../fig/#1}

\newcommand{\bc}{\begin{center}}
\newcommand{\ec}{\end{center}}

\newcommand{\bdm}{\begin{displaymath}}
\newcommand{\edm}{\end{displaymath}}

\newcommand{\beq}{\begin{equation}}
\newcommand{\eeq}{\end{equation}}

\newcommand{\bfl}{\begin{flushleft}}
\newcommand{\efl}{\end{flushleft}}

\newcommand{\bt}{\begin{tabbing}}
\newcommand{\et}{\end{tabbing}}

\newcommand{\beqn}{\begin{align}}
\newcommand{\eeqn}{\end{align}}

\newcommand{\beqs}{\begin{align*}} % no equation numbers
\newcommand{\eeqs}{\end{align*}}  % no equation numbers

%% macros for theorem-like environments
\newtheorem{assumption}{Assumption}

%\newtheorem{theorem}{Theorem}
% \newtheorem{proof}{Proof}

%% AMSA theorem
\ifamsa
    \newtheorem{condition}{Condition}
    \newtheorem{rul}{Rule}
    \newtheorem{definition}{Definition}
    \newtheorem{corollary}{Corollary}
    \newtheorem{remark}{Remark}
    \newtheorem{lemma}{Lemma}
    \newtheorem{proposition}{Proposition}
    \newtheorem{example}{Example}
\fi

%\newtheorem{example}[remark]{Example}



 % put all macros there
\usepackage{amsmath}
%DIF PREAMBLE EXTENSION ADDED BY LATEXDIFF
%DIF UNDERLINE PREAMBLE %DIF PREAMBLE
\RequirePackage[normalem]{ulem} %DIF PREAMBLE
\RequirePackage{color}\definecolor{RED}{rgb}{1,0,0}\definecolor{BLUE}{rgb}{0,0,1}


\begin{document}

\title{Response to the Two Reviewers}
\vspace{5mm}

We would like to thank the associate editor for the time and efforts in putting our manuscript through the review process. We would also like to thank the two anonymous referees for their careful reviews and constructive comments. The concerns of the reviewers have been addressed point by point. We are resubmitting the revised manuscript for review. 

Let us summarize the major changes in this new revision:
\begin{itemize}
\item We corrected all of the typos based on the first reviewer's comments. 
\item In response to the second reviewer's criticisms, we emphasized on the new algorithms and new applications  in our article and, in addition,  added a new primal-dual coordinate-update algorithm in Section 4.4. We provided the convergence proof for the added algorithm   in Appendix D.
\item We updated a  numerical experiment in Section 6.2 to demonstrate the superiority of coordinate friendly methods over the start-of-the-art method. 
\item We edited the introduction to reflect the changes made, as well as the other contributions of our article.
\end{itemize}

\section{Response to reviewer 1}
``This paper proposes a high-level abstraction for coordinate update method in optimization. The concept of ``coordinate friendly operator'' is introduced and the proposed framework unifies many previously studied coordinate update type methods in the literature."

``In general, the paper is well written and well organized. But this reviewer still finds some typos and awkward statements. In the following, I will list some of the typos that I found. The authors are suggested to do more thorough and careful checking to make sure that there is no more typos."

\rev{\textbf{Response:} Thanks for your careful review. We have read through the paper multiple times and improved some then awkward statements. We have also corrected the typos, one by one, in the new manuscript. Thanks again. 
}

\section{Response to reviewer 2}
``This paper discusses coordinate update methods, which are useful for solving
large-scale problems that admit a fixed-point formulation, including various
coordinate descent methods in optimization, solution of linear systems of
equations, and many operator splitting algorithms. In particular, the authors
introduce several notions of Coordinate Friendly (CF) operators, discuss
composite and combination of CF operators, use them to characterize several
classical and recent operator splitting schemes, and also obtain some new
coordinate update methods. These methods are illustrated by examples from
machine learning, imaging, finance, and distributed computing, among others."

\rev{\textbf{Response:} Thanks for the summary.

Before we respond to other comments from the reviewer, we would like to share with the reviewer our motivation for our work. 

Coordinate update methods have had very exciting progress recently. Nonetheless, most recent papers focus on a few classes of important applications such as regularized empirical risk minimization. Possessing a great power,  it would be a pity if coordinate update methods are not applied to  more problems and show strong performance. Clearly,  strong performance of  coordinate update methods relies on solving simple subproblems. However,  it was not always easy  to get simple subproblems and  implement their solvers properly. Even for domain experts, when they face problems combining constraints,   objective functions, both separable and non-separable terms, and/or both convex and nonconvex parts, those questions are nontrivial to answer. 

Clearly, not all problem structures and algorithms are amenable to coordinate update. Therefore, we need a qualifier, a simple condition that needs to be satisfied for efficient coordinate update. That is the coordinate-friendly property we propose in our article. 

The techniques to obtain simple subproblems, develop their solvers, and qualify them for coordinate update are crucial to the successful applications of coordinate update methods, which are the focus of our article.
}


\textbf{Comment 1:} The focus of this paper is on summarizing the components and composition of
efficient coordinate-update methods, and largely ignore convergence guarantees.

\rev{\textbf{Response:}  We agree that the convergence guarantees were ignored in our article on purpose. In the new revision, we added the convergence guarantee for one of the novel algorithms in the paper. It is an algorithm performing async-parallel primal-dual coordinate update, capable of solving problems such as second-order cone programs by multiple computers simultaneously performing coordinate updates. To avoid a sharp increase in technicality level, we have added the proof to the Appendix. Due to the focus and the page limit, we have to ignore the convergence guarantees of other new algorithms.

%In fact, the goal of this paper is to identify the favorable structures of an operator which make coordinate-update computationally worthy.  As for convergence guarantees, in this revision, we proved the convergence of our primal-dual coordinate-update schemes under the async-parallel algorithmic framework (which includes the stochastic coordinate selection rule as a special case). The convergence of other coordinate-update schemes is omitted due to page limit, but they all converge under the async-parallel algorithmic framework according to the previous work \cite{Peng_2015_AROCK}.
}

\textbf{Comment 2:} On one hand, this seems to be a timely topic given the recent flurry of
research activities on coordinate update methods and their applications, and
the materials presented may help the understanding of the common theme of
these methods, especially for practitioners. 

\rev{\textbf{Response:} We agree with the reviewer that the target audiences of our paper are the practitioners. 

We would like to emphasize that those coordinate-update algorithms obtained after the Douglas-Rachford splitting or primal-dual splitting are new and uncommon. Those algorithms embrace two different sets of techniques, coordinate update and operator splitting, for challenging applications.  %were never proposed to solve them, yet our numerical experiments show that they have strong performance. One of the aims of our paper is to bridge the gap between coordinate-update theory and the applications.
}

\textbf{Comment 3:} But on the other hand, I feel
there is not enough innovative ideas other than summarizing several well-known
structure of coordinate update methods, and the technical depth of this paper
does not reach the high-quality of a first-class journal on mathematical
sciences. 

\rev{\textbf{Response:} Our exposition includes both well--known  and \emph{new} structures. Several composite structures are new, and their integration with operator splitting such as the proposed primal-dual coordinate-update algorithm are also new. We have systematically reasoned that certain quantities must be cached and maintained in memory, or coordinate update would not be efficient. These expositions lead to new coordinate-update algorithms for second-order cone programming, nonnegative matrix factorization, CT image processing, and portfolio optimization, which were not previously treated with similar coordinate updates. We also get new coordinate-update algorithms for overlapping group LASSO. 

Considering the audience and goal of this article, we intentionally keep its technical depth at a low--medium level so that non domain experts can understand this article. Operator splitting and coordinate descent convergence can go very sophisticated, fending off potential users. Hence, we decide to leave them out in order to promote applications. 

While it will take time to see where the journal ``Annals of Mathematical Sciences and Applications" ends up settling down with its emphases, we would like to test the water by aiming at successful \emph{applications} of coordinate update methods. 

%There are many innovative ideas in this paper. First, we identified and classified coordinate friendly structures.  Formal CF analysis, especially that for composite operators, has never appeared in the literature, although it appears to be simple to understand. Second, we apply our new coordinate-update algorithms to problems that were never treated with coordinate update methods before, for example, second-order cone programming, nonnegative matrix factorization, image processing, portfolio optimization, etc. Third, we also provide new coordinate-update approaches for problems which were treated before with coordinate-update methods, for instance, support vector machines and group Lasso. Forth, our overlapping-block coordinate update scheme for primal-dual algorithms has never appeared in the literature.

%For the first submission, we intentionally reduced the technical depth of this paper for the following reason. We realize the big gap between coordinate-update theory and the applications in the current literature. Most of the existing coordinate-update papers only consider the several limited well known problems which have separable structures. Many other applications can actually be solved with the more efficient coordinate update methods as demonstrated in this paper. 

%Based on our understanding, the aim and scope of this first class journal is not just about mathematical science, but also applications.  In order to explain how to apply coordinate method to more complicated applications and expand their influences, we put less emphasis on the technical depth.
}

\textbf{Comment 4:} From a more application oriented view, there is no sufficient
justification (either theoretical or empirical) for the efficiency of the
derived algorithms, nor comparison with state of the arts for the particular
applications discussed. Thus it is not clear how useful they will become,
other than illustrating some general ideas.

\rev{\textbf{Response:} We respectfully disagree with the reviewer for the following reasons. 

The original submission of this article does include the numerical results for three different coordinate-update algorithms, which are applied to three different problems. We compared the performance of our coordinate-update algorithms with the corresponding full-update algorithms, which are themselves state-of-the-art methods. The numerical experiments showed the advantage of coordinate-update algorithms. In the third numerical experiment, we compared an \emph{async}-parallel coordinate-update algorithm with the corresponding  \emph{sync}-parallel one and observed significant speedups. 

On the other hand, we do agree that the original submission ignored theoretical justification (as explained in our last response).

In this revision, we updated our second numerical example to a new instance of CT image. The coordinate-update algorithm continues to show its superiority in efficiency.

We also added an analysis for the convergence of one of our primal-dual coordinate-update algorithms (the most sophisticated one, under the async-parallel setting).

Hopefully these numerical and theoretical results will convince the reviewer and the readers that coordinate update algorithms are worth serious consideration.}
%\bibliographystyle{plain}
%\bibliography{response0}
\end{document}

% !TEX root = ./main.tex
\section*{Notation}

\begin{table}[htbp]
 \caption{\label{tab:notation}}
 \begin{tabular}{ll}
  \toprule
  Notation & Explanation \\
  \midrule
  $\HH,\GG$ & Hilbert spaces ($\HH$ for primal, $\GG$ for dual) \\
  %& dual Hilbert space \\
  $\HH_i,\GG_i$ &``smaller'' Hilbert spaces \\
  % &``smaller'' dual Hilbert space \\
  $\II := \{i_1, ..., i_m \} \subseteq \{1, .., n\}$  & subset of coordinate indexes\\
  $X, Y$ & feasible sets \\
  \midrule
  $\cS, \cT, \cA, \cB$ & operators \\ 
  $\cI$ & identity map \\
  $\cU$ & sampling operator \\
  $\cJ_{\cT} := (\cI + \cT) ^ {-1}$ & resolvent of $\cT$ \\ 
  $\cR_{\cT} := 2 \cJ_{\cT} - \cI$ & reflection of $\cT$ \\ 
  $\cT_{\alpha} := (1 - \alpha) \cI + \alpha \cT$ & $\alpha$-averaged operator\\
%  $\cS := \cI - \cT$ & auxiliary operator \\
  $\TFBS$ & forward-backward splitting operator \\
  $\TBFS$ & backward-forward splitting operator \\
  $\TDRS$ & Douglas-Rachford splitting operator \\
  $\TPRS$ & Peaceman-Rachford splitting operator \\
  %$\cT_{\text{RPRS}}$ & Relaxed Peaceman-Rachford splitting operator \\
  $\TFBFS$ & forward-backward-forward splitting operator \\
  $\TFDRS$ & forward-Douglas-Rachford splitting operator \\
  
  \midrule
  $x, y$ & primal variables \\
  $s, t$ & dual variables \\
  $z$ & aggregation of variables \\ 
  \midrule
  $\eta, \gamma$ & step sizes\\
  $L$ & Lipschitz constant\\
  $\alpha$ & $\alpha$-average constant \\
  $\beta$ & cocoercive constant \\
  $\lambda$ & relaxation parameter \\
  $m$ & number of coordinates for primal variable\\
  $n$ & dimension of a finite primal dimensional space \\
  $p$ & number of coordinates for dual variable\\ 
  $q$ & dimension of a finite dual dimensional space \\
  \midrule
  $I$ & identity matrix \\
  $A, B$ & matrix \\
  $M$ & metric for primal-dual methods \\
  \midrule
  $\circ$ & composition\\
  $\times$ & Cartesian product \\
  $(\cdot)^k$ & superscript $k$ as iteration counter \\
  $(\cdot)_i$ & subscript $i$ as coordinate index\\
  \bottomrule
 \end{tabular}
\end{table}


%\end{section}
